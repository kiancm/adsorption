\documentclass{article}

\usepackage[style=ieee]{biblatex}
\usepackage{setspace}

\author{Kian Mehrabani}
\date{April 29, 2021}
\title{Modeling Adsorption Isotherms with Grand Canonical Monte Carlo Simulation Techniques}

\bibliography{adsorption}

\begin{document}

\maketitle\newpage
\doublespacing

\section{Introduction}
Adsorption is a popular field of research with important applications in catalysis, energy storage systems, and nuclear waste sequestration \cite{h2-storage,sbmof-discovery}.
Key research objectives in this space involve materials discovery to optimize adsorption performanc metrics such as adsorption loading at specified conditions and selectivities to target adsorbates.
With the advent of modern computing resources, it is now possible to develop new material structures with the assistance of data from adsorption process simulations and performance indicators form large-scale isotherm data analysis.
In this work, the adsorption isotherm of xenon gas on a novel metal-organic framework (MOF) will be simulated using Grand Canonical Monte Carlo (GCMC) simulation methods,
then compared with experimental data and a classical adsorption isotherm model.

\section{Background}

\end{document}
